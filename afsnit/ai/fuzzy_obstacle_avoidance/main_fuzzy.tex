\documentclass[../../../Main.tex]{subfiles}
\begin{document}

In this section the implementation of obstacle avoidance using fuzzy control will be described.\\

Requirements
\begin{itemize}
	\item Avoid simple local obstacles while maintaining course towards
		the robots short-term goal.
\end{itemize}

\subsubsection{Design}%
\label{ssub:design}

The primary focus of the fuzzy controller is that it does not make sense to avoid an
obstacle that is to the left when going straight. To accommodate it makes sense to look at
a rectangle with a width equal or greater than the width of the robot. The controller also needs
to be able to steer towards a goal, so the controller should also needs to be aware of obstacles
in its immediate vicinity to determine whether it is safe to turn towards its goal.
The described zones can be seen in figure \ref{fig:ai:avoidance_area}.


\begin{figure}[h]
	\centering
	\def\svgwidth{0.2\textwidth}
	\includesvg[\main/afsnit/ai/fuzzy_obstacle_avoidance/img/]{avoidance_area}
	\caption{Zones of interest for obstacle avoidance with coordinate system.
	Blue is the immediate obstacle avoidance zone.
	Red is the implicit avoidance zone.}
	\label{fig:ai:avoidance_area}
\end{figure}


\subsubsection{Implementation}%
\label{ssub:implementation}

\paragraph{Preprocessed input data}%
\label{par:preprocessed_input_data}
The lidar data return polar coordinates. 
These are transformed into the cartesian coordinates seen in figure \ref{fig:ai:avoidance_area}.

\paragraph{Fuzzy input variables}%
\label{par:fuzzy_input_variables}
\begin{itemize}
	\item \textit{Distance} \\
		\textit{Distance} is the smallest y coordinate in the interval $0\si{m} \leq y \leq 1\si{m}$.\\
		\textit{Distance}  also implements the following terms:
		\begin{itemize}
			\item Close\\
				a ramp function got from (0,1) to (0,1,).
			\item Far\\
				a ramp function going from (0,0) to (1,1).
		\end{itemize}

	\item \textit{Obstacle} \\
		\textit{Obstacle} is the total center of mass of any objects in the
		immediate avoidance zone (see figure \ref{fig:ai:avoidance_area}).\\
		\textit{Obstacle} makes use of the following terms:
		\begin{itemize}
			\item Left\\
				a ramp function got from (0,1) to (0,1,).
			\item Right\\
				a ramp function going from (0,0) to (1,1).
		\end{itemize}
	\item \textit{ErrorYaw} \\
		\textit{ErrorYaw} is the difference between the robots global orientation from the orientation
		required for the robot to look straight at its goal.\\
		\textit{ErrorYaw} makes us of the following terms:
		\begin{itemize}
			\item Left\\
				a ramp function going from (0,0) to (3.15,1).
			\item Right\\
				a ramp function going from (-3.15,1) to (0,0).
			\item Center\\
				a triangle function going from (-1,0) to (0,1) to (1,0).
		\end{itemize}
\end{itemize}



%\begin{figure}[h]
%	\centering
%	\def\svgwidth{\textwidth}
%	\includesvg[\main/afsnit/ai/fuzzy_obstacle_avoidance/img/]{lidar_sensor}
%	\caption{lidarsen}
%	\label{fig:ai:lidar_sensor_format}
%\end{figure}

\paragraph{Fuzzy output variables}%
\label{par:output_variables}
\begin{itemize}
	\item \textit{Velocity} \\
		\textit{Velocity} is the velocity of the robot.
	\item \textit{mSteer} \\
		\textit{mSteer} is the angular acceleration of the robot.
\end{itemize}

\paragraph{Rule base}%
\label{par:rulebase}

To avoid a weird wiggle while driving along obstacles that intersects with the direct path to goal, 
2 rule bases are used.\\
The rule base seen in listing \ref{lst:ai:obsav} is responsible for obstacle avoidance and is not
active when within 1 meter of the goal.\\
The rule base seen in listing \ref{lst:ai:marbcatch} is responsible for steering towards the goal
and it is active when nothing is in the implicit avoidance zone and nothing is in the
immediate avoidance zone.\\
Rule base 1 is active when the robot is not in the vicinity of its goal.
Rule base 2 is active when rule base 1 is not or when there is no obstacle in the implicit
avoidance zone.

\lstinputlisting[
	caption=Rulebase for obstacle avidance,
	label=lst:ai:obsav,
	firstline=49,
	lastline=52]{%
	\main/afsnit/ai/fuzzy_obstacle_avoidance/rulebase.fll} 

\lstinputlisting[
	caption=Rulebase for going towards goal,
	label=lst:ai:marbcatch,
	firstline=60]{%
	\main/afsnit/ai/fuzzy_obstacle_avoidance/rulebase.fll} 

\subsubsection{Tests}%
\label{ssub:activation}

\begin{figure}[h]
  \centering
	\makebox[\textwidth][c]{
		\subfloat{
			\def\svgwidth{0.6\textwidth}
			\includesvg[\main/afsnit/ai/fuzzy_obstacle_avoidance/img/]{ObsDisSpeed}
		}
		\subfloat{
			\def\svgwidth{0.6\textwidth}
			\includesvg[\main/afsnit/ai/fuzzy_obstacle_avoidance/img/]{ObsDisSteer}
		}
	}
  \caption{Activation degree for obstacle avoidance rule base}
	\label{fig:fuzzy:activation}
\end{figure}

As can be seen in figure \ref{fig:fuzzy:activation} the exact location of the obstacle does not
affect the speed of the robot only the distance. Likewise is the steer variable not affected by
the distance. This should also be apparent from the rule base in listing \ref{lst:ai:obsav}.


\begin{figure}[H]
  \centering
	\makebox[\textwidth][c]{
		\subfloat{
			%\def\svgwidth{0.6\textwidth}
			\includesvg[\main/afsnit/ai/fuzzy_obstacle_avoidance/img/]{SpeErr}
		}
		\subfloat{
			%\def\svgwidth{0.6\textwidth}
			\includesvg[\main/afsnit/ai/fuzzy_obstacle_avoidance/img/]{SteErr}
		}
	}

	\subfloat{
		\def\svgwidth{0.6\textwidth}
		\includesvg[\main/afsnit/ai/fuzzy_obstacle_avoidance/img/]{SteSpeErr}
	}
  \caption{Activation degree for marble catcher rule base}
	\label{fig:fuzzy:activationMarb}
\end{figure}

In figure \ref{fig:fuzzy:activationMarb} the activation degree for  the marble catcher rule base 
can be seen. The two rule bases are displayed independently of each other as they are never active at the same time.

\subsubsection{Discussion \& Conclusion}%
\label{ssub:discusion}


A weakness of this controller is if there is an obstacle within 1 meter of the short-term goal of
the robot as this would obstacle would not be detected and would therefore not be avoided.\\
An improvement could be to look in the direction of the goal and see if there is an obstacle that
needs to be avoid, similar to the bug3 algorithm.



\end{document}
