\documentclass[../../../main.tex]{subfiles}

\begin{document}

\subsection{Discussion}
\label{sec:cv_discussion}
Given the algorithms implemented and described in section \ref{sec:cv_cont_bgp_impl}, quality of results together with improvement possibilities are discussed in this section.\\

% Quality of data - What can be seen and deducted
To first consider the results shows in section \ref{sec:cv_hough_impl}, 
As expected the marbles being different colors are not the attribute effecting marble detection the most, instead it is the contrast and with great certainty marbles can now be found.\\
As a non optimal attribute of the algorithms used a small uncertainties occur do to the instability in the Hough transform.\\
To solve this problem and optimize performance a combination of the methods described in section \ref{sec:cv_cont_bgp_impl} and the previous could be applied for future implementation.\\
The method here would determine the location of the marble with the use of \texttt{HoughCircles()}, thus center and radius are determined.
Based then on radius and center a ROI i chosen within the found circle. The found ROI would then be used for background projection, 
thus being able to detect the actual colors within the marble. Having done this the image can be converted to binary and a contour can be found.\\
By applying this method marble detection would be independent of marble colors, while having the contour accuracy.\\
If further precision would be needed, camshift can be added by applying the OpenCV function \texttt{CamShift()}. 
By passing \texttt{CamShift()} the previous frame, the current frame and some ROI for analysis, movement can be compensated for such that a more smooth
following of the marbles when the robot i moving is achieved.\\ 

\end{document}
