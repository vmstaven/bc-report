\documentclass[../../../main.tex]{subfiles}

\begin{document}

\subsection{Introduction}
\label{sec:datapp_introduction}

In this chapter, the goal is to preprocess the data available to a format usable for 
semantic segmentation using a given deep learning algorithm.

The data provided for this task was collected from a flying unmanned aerial vehicles (UAV) and comes in the form
of a RPG image and a Digital Surface Models (DSM). The RGB image is a orthomosaic, generated by the UAV, and the DSM holds information 
about the height of the terrain in a given pixel. The dimensions of these images range from 5688 by 6430 to 20672 by 13788 pixels,
and thus these are too large for direct use as input for the AI.  
The desired dataformat for the AI is images of significant smaller proportions, 256 is chosen as the proper dimensions for the tiles,
as this is commonly used.
The pipeline developed for converting the orthomosaic and DSM to a useable dataformat is as follows.

First the dataset is parsed through a MATLAB script,


The goal is thus to place all the tree 
% What do we have? and what do we want?

% how are we gonna achieve this?


\end{document}
